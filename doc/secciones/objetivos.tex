% OBJETIVOS

\newpage
\chapter{Objetivos}
\label{objetivos}

\section{Objetivo general}

Este proyecto es una propuesta del grupo de investigación de la Universidad de Burgos, para analizar ramas del aprendizaje semi-supervisado que no se habían contemplado hasta el momento. El objetivo general del presente trabajo es realizar una implementación y validación de algunos métodos de aprendizaje semi-supervisado, centrándose en los ámbitos de grafos y árboles y compararlos con otros algoritmos bien afianzados como RandomForest, Self-Training o Co-Forest. 


\section{Objetivos específicos}

\begin{enumerate}[label=\destacado{\arabic*.}]
  \setlength\itemsep{0.5em}
  \item Realizar una revisión bibliográfica (guiada\footnote{Gracias a información previa recopilada por el grupo de investigación de la Universidad de Burgos.}) para establecer los métodos más prometedores y útiles.
  \item Implementar desde cero un algoritmo de construcción de árboles que permita trabajar con datos etiquetados y no etiquetados (semi-supervisado).
  \item Implementar desde cero varios algoritmos de construcción de grafos y de \textit{label propagation} pues debido a la naturaleza de los algoritmos basados en grafos, necesitan de dos pasos separados (construcción de grafo y propagación de etiquetas).
  \item Seleccionar los conjuntos de datos adecuados para la experimentación de estos algoritmos para una validación exhaustiva que refleje su utilización en muy diversos ámbitos. Al menos, 20 \textit{datasets}.
  \item Codificar experimentos adecuados para los algoritmos. Previsiblemente incluirá: preprocesado de datos, codificación de validaciones cruzadas, experimentación de parámetros específicos (influencia) o graficar resultados, entre otros.
  \item Comparar los algoritmos desarrollados con algoritmos afianzados del estado del arte para la extracción de resultados y conclusiones.
\end{enumerate}

En las etapas finales del proyecto, una vez analizada la información obtenida, se consideró en conjunto con el grupo de investigación de la Universidad de Burgos, crear una página web con el propósito de permitir la visualización de los algoritmos de grafos desarrollados. En el Anexo \ref{apendice-b} se muestra la documentación de la aplicación web.



