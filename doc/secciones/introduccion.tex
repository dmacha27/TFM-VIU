% INTRODUCCIÓN

\newpage
\chapter{Introducción}
\label{introduccion}

El aprendizaje automático o \textit{machine learning} como disciplina de la inteligencia artificial resulta ser uno de los campos más cotizados y que despierta más interés en prácticamente cualquier aplicación (investigación, automatización, sistemas de ayuda, detección...). Existe una división muy clara del aprendizaje automático que consta de: aprendizaje supervisado y el no supervisado. Pero existe otra división que no suele mencionarse, y que puede ser muy beneficiosa, este es el aprendizaje semi-supervisado.

De forma resumida, el aprendizaje supervisado trata de aprender de datos de los que se sabe lo que representan para después poder inferir este conocimiento para nuevos datos (por ejemplo, dadas las características de una flor, se intenta predecir de qué clase concreta es), el aprendizaje no supervisado trata de aprender de datos de los que \textbf{no} se sabe lo que representan, se utiliza en tareas en las que es necesario realizar agrupaciones o divisiones con base en las similitudes/disimilitudes de los ejemplos (es decir, podría distinguir entre animales que tienen plumaje de los que no sin tener el conocimiento de qué animales son concretamente). En el caso del aprendizaje supervisado, el etiquetado de los datos suele ser un proceso costoso (es posible imaginar, por ejemplo, la cantidad de tiempo y recursos que podría suponer el etiquetado masivo de millones de muestras de posibles cánceres). En la realidad, la mayor parte de los datos no están etiquetados. Ante esta necesidad aparece el aprendizaje semi-supervisado, que se sitúa en la intersección del supervisado y no supervisado y que permite aprovechar los escasos datos etiquetados para inferir su conocimiento a los no etiquetados.

Como se ha comentado, los algoritmos semi-supervisados suponen un área de mucha utilidad dentro del \textit{machine learning}, sin embargo, así como para otras ramas (como el aprendizaje supervisado y no supervisado) es posible encontrar numerosas bibliotecas y algoritmos bien desarrollados y probados, para el semi-supervisado todavía hay una gran cantidad de investigación que no se ha materializado (o que si lo ha hecho, no se ha publicado). Se pretende contribuir en el desarrollo de estos algoritmos. Para ello, el objetivo consiste en realizar una revisión bibliográfica de métodos semi-supervisados con la posterior implementación de algunos de los más prometedores junto con su comparación exhaustiva con otros métodos afianzados.

La organización de la documentación es la siguiente: se comienza con el estado del arte para tener un contexto de panorama de los métodos implementados, se introduce el marco teórico del aprendizaje automático, de los árboles de decisión y los grafos, se plantean los objetivos del proyecto alineados con los propuestos por la Universidad de Burgos y se expone el desarrollo completo de los métodos seleccionados, experimentos, resultados y conclusiones. Por último, se analizan las posibles perspectivas de futuro.
\clearpage