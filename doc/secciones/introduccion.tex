% INTRODUCCIÓN

\cleardoublepage

\chapter{Introducción}
\label{introduccion}

El aprendizaje automático o \textit{machine learning} como disciplina de la inteligencia artificial resulta ser uno de los campos más cotizados y que despierta más interés en prácticamente cualquier aplicación (investigación, automatización, sistemas de ayuda, detección...). Existe una división muy clara del aprendizaje automático que consta de: aprendizaje supervisado y el no supervisado. Pero existe otra división que no suele mencionarse, y que puede ser muy beneficiosa, este es el aprendizaje semi-supervisado.

De forma resumida, el aprendizaje supervisado trata de aprender de datos de los que se sabe lo que representan para después poder inferir este conocimiento para nuevos datos (por ejemplo, dadas las características de una flor, se intenta predecir de qué clase concreta es), el aprendizaje no supervisado trata de aprender de datos de los que \textbf{no} se sabe lo que representan, se utiliza en tareas en las que es necesario realizar agrupaciones o divisiones en base a las similitudes/disimilitudes de los ejemplos (por ejemplo, podría distinguir entre animales que tienen plumaje de los que no sin tener el conocimiento de qué animales son concretamente). En el caso del aprendizaje supervisado, el etiquetado de los datos suele ser un proceso costoso (es posible imaginar, por ejemplo, la cantidad de tiempo y recursos que podría suponer el etiquetado masivo de millones de muestras de posibles cánceres). En la realidad, la mayor parte de los datos no están etiquetados. Ante esta necesidad aparece el aprendizaje semi-supervisado, que se encuentra a caballo entre el supervisado y no supervisado y que permite aprovechar los escasos datos etiquetados para inferir su conocimiento a los no etiquetados.

\section{Aprendizaje automático}

El aprendizaje automático (\textit{machine
learning}) según~\cite{intelligent:ml} es una rama de la Inteligencia Artificial y se trata de una técnica de análisis de datos que enseña a las computadoras a aprender de la \textbf{experiencia} (como los humanos). Para llevar a cabo este proceso, el aprendizaje automático requiere de una amplia cantidad de datos, o los necesarios para el problema específico en cuestión. Estos datos son procesados mediante algoritmos, los cuales se alimentan de ejemplos (también conocidos como instancias o prototipos). A través de estos ejemplos, los algoritmos tienen la capacidad de generalizar comportamientos ocultos.

Estos algoritmos mencionados mejoran su rendimiento iterativamente y de forma automática durante su entrenamiento e incluso también durante su aprovechamiento/explotación. El aprendizaje automático ha adquirido una gran relevancia en una amplia variedad de áreas como la visión artificial, automoción, detección de anomalías o automatización, entre otras. El aprendizaje automático generalmente se clasifica en tres tipos: aprendizaje supervisado, aprendizaje no supervisado y aprendizaje por refuerzo. Sin embargo, ha surgido una nueva disciplina que se sitúa entre el aprendizaje supervisado y el no supervisado, utilizando tanto datos etiquetados como no etiquetados durante el proceso de entrenamiento~\cite{vanEngelen2020}.

En la figura~\ref{fig:figuras/taxonomia.png} se presenta una clasificación del aprendizaje automático.

\imagen{figuras/taxonomia.png}{Clasificación de aprendizaje automático}{Clasificación de aprendizaje automático, basado en~\cite{neova:taxonomy}.}{1}

\subsection{Aprendizaje supervisado}
\label{aprendizaje-supervisado}

Los algoritmos de aprendizaje supervisado utilizan datos etiquetados durante su proceso de entrenamiento~\cite{david:sl}. Un ejemplo popular de datos etiquetados podría ser un conjunto flores de iris y las posibles etiquetas podrían ser: setosa, versicolor y virginica. Estos datos estarán formados por un conjunto de características (en el caso de las flores de iris podrían ser la longitud y ancho del sépalo y del pétalo). Estas características podrían ser categóricas, continuas o binarias~\cite{salim:sl}.

Para generar un modelo correcto, estos datos son divididos en varios subconjuntos: conjunto de entrenamiento (\textit{training data set}), conjunto de validación (\textit{validation data set}) y conjunto de test (\textit{test data set}). El conjunto de entrenamiento corresponde con la porción de los datos que el algoritmo utilizará para aprender un modelo que generalice los patrones ocultos subyacentes. El conjunto de validación permite comprobar, durante el proceso de entrenamiento, que el modelo que se está generando no memoriza los datos (fenómeno conocido como sobreajuste), también sirve para finalizar el entrenamiento (e.g. el error en validación aumenta durante varias iteraciones). Una vez que el algoritmo ha generado un modelo, se utiliza el conjunto de test para comprobar el rendimiento real (una estimación)~\cite{enwiki:conjuntos}. Ningún dato de este último conjunto ha sido ``visto'' por el modelo previamente.


En la figura~\ref{fig:figuras/AprendizajeSupervisado.PNG} se encuentra un diagrama con el funcionamiento general.

\imagen{figuras/AprendizajeSupervisado.PNG}{Funcionamiento general del aprendizaje supervisado}{Funcionamiento general del aprendizaje supervisado, basado en~\cite{salim:sl}.}{1}

Partiendo del concepto de etiqueta de un dato, el problema será de \textbf{clasificación} si los valores que puede tomar la etiqueta representan un conjunto finito. Por otro lado, si estos valores son continuos, el problema será de \textbf{regresión}.

\begin{itemize}
    \item \textbf{Clasificación}: Un modelo entrenado en un problema de clasificación se denomina clasificador. Ante un nuevo dato, el clasificador predecirá su etiqueta correspondiente. Por lo general, a cada valor de etiqueta se le suele llamar clase. Dependiendo de la cantidad de valores, se referirá a un problema binario o multiclase.

    \item \textbf{Regresión}: En este caso, ante un nuevo dato, el modelo predecirá un valor continuo. La idea subyacente es evaluar una función (ajustada/aprendida durante el entrenamiento) dado un dato como variables de entrada.

\end{itemize}


\subsection{Aprendizaje no supervisado}
\label{aprendizaje-no-supervisado}

A diferencia del aprendizaje supervisado, el no supervisado no trabaja con datos etiquetados y clases. Según~\cite{salim:usl} esto quiere decir que nosotros no ``supervisamos'' el algoritmo. No se le añade ese conocimiento extra. Estos algoritmos intentarán descubrir patrones que se encuentren en la propia estructura de los datos (de sus características). La idea del aprendizaje no supervisado es estudiar las similitudes/disimilitudes que hay entre los datos y, por ejemplo, obtener una separación o agrupación de los mismos (e.g. separación de especies en imágenes de animales sin conocer el animal concreto).
 
Entre las principales aplicaciones del aprendizaje no supervisado se encuentran las siguientes:
\vspace{-4px}
\begin{enumerate}
    \item \textbf{Agrupamiento (Clustering)}: Divide los datos en grupos. Los ejemplos de un grupo tendrán cierta similitud entre ellos, mientras que todos los ejemplos de ese grupo serán disimilares a los de otro grupo (y por eso se generó esa división). Algunos algoritmos necesitan conocer de antemano el número de grupos en los que dividir lo datos, otros son capaces de descubrir cuántos grupos existen~\cite{salim:usl}.
    \item \textbf{Reducción de la dimensionalidad}: Los conjuntos de datos generalmente tiene un número bastante grande de características. Esto hace que los algoritmos de aprendizaje sean más lentos. La reducción de dimensionalidad hace referencia a la reducción de número de características tratando de no perder información al hacerlo.
    Según
   ~\cite{javatpoint:reduccionsdims} se denomina
    como: \begin{quote}<<\textit{Una forma de convertir conjuntos de datos de alta dimensionalidad en
    conjunto de datos de menor dimensionalidad, pero garantizando que proporciona
    información similar.}>>\end{quote} 
    
    Algunos ejemplos concretos de reducción de dimensionalidad son:
    \begin{itemize}
        \item Análisis de Componentes Principales (PCA).
        \item Cuantificación vectorial.
        \item Autoencoders.
    \end{itemize}
\end{enumerate}

\imagenconurl{figuras/Clustering.jpg}{Clusters}{\footnotesize{\emph{Clusters}. A la izquierda los datos sin agrupar y a la derecha los datos coloreados según la pertenencia a los distintos grupos. By
hellisp - Own work, Public Domain,
\url{https://commons.wikimedia.org/w/index.php?curid=36929773}. }}{0.7} 

\subsection{Aprendizaje semi-supervisado}
\label{aprendizaje-semi-supervisado}

Según~\cite{vanEngelen2020}, el aprendizaje semi-supervisado es la rama del
aprendizaje automático que utiliza tanto datos etiquetados como no etiquetados 
durante el entrenamiento. Es por esto que se dice que está a medio camino entre 
el aprendizaje supervisado y el no supervisado. Como se ha comentado, el problema
al que todos los algoritmos se enfrentan en la realidad es a la escasez de datos 
etiquetados, pues es un proceso costoso. Gracias a la naturaleza del semi-supervisado,
hace que sea una buena aproximación para esos casos. Por lo general, suele aplicarse 
en problemas de clasificación.

\subsubsection{Suposiciones}
¿Y por qué utilizar aprendizaje semi-supervisado? Lo cierto es que algunos
de los algoritmos existentes de aprendizaje supervisado funcionan bastante bien
incluso con pocos datos etiquetados. Sin embargo, los datos no etiquetados podrían
aprovecharse para mejorar el rendimiento.

El objetivo, por tanto, del aprendizaje semi-supervisado será obtener clasificadores
que obtengan mejores resultados que los del aprendizaje supervisado. En~\cite{vanEngelen2020}
se especifican unas condiciones que han de cumplirse.

La primera premisa que se debe cumplir es que la distribución $p(x)$ de entrada contenga
información sobre la distribución posterior $p(y|x)$~\cite{vanEngelen2020}.

\begin{mainbox}{Smoothness assumption}
    Probablemente, si dos ejemplos se encuentran próximos en el espacio, comparten
    la misma etiqueta.
\end{mainbox}

\medskip

\begin{mainbox}{Low-density assumption}
    La frontera de decisión en un problema de clasificación se encontrará en una zona del espacio
    en el que existan pocos ejemplos.
\end{mainbox}

\medskip

\begin{mainbox}{Manifold assumption}
    Los ejemplos suele encontrarse en una estructuras de dimensionalidad baja (algunas características
    no son útiles), denominadas \emph{manifolds}. Los ejemplos que se encuentren en una misma \emph{manifold} comparten la misma etiqueta~\cite{towardsdatascience:semi,vanEngelen2020}.
\end{mainbox}

\medskip

\begin{mainbox}{Cluster assumption}
    Los ejemplos que se encuentren en un mismo grupo compartirán la misma etiqueta.
\end{mainbox}


El concepto clave de todas estas suposiciones es el de la ``similitud'' 
(ejemplos próximos en el espacio, ejemplos en misma manifold, mismo grupo...). 
Es por esto que la \textit{Cluster assumption} es una generalización del
resto (o el resto son versiones de esta).

Por esto, para que el que \destacado{el aprendizaje semi-supervisado mejore al supervisado}
es necesario que se cumpla dicha suposición generalizada. Si no fuese así (i.e. datos no agrupables),
el aprendizaje semi-supervisado no mejorará al supervisado~\cite{vanEngelen2020}.


En la figura~\ref{fig:figuras/AprendizajeSemiSupervisado.pdf} se presenta la taxonomía general del aprendizaje semi-supervisado.

\imagen{figuras/AprendizajeSemiSupervisado.pdf}{Taxonomía de métodos semi-supervisados}{Taxonomía de métodos semi-supervisados~\cite{vanEngelen2020}.}{1}

Sin pérdida de generalidad, este trabajo estará centrado en métodos semi-supervisados basados en grafos y árboles (intrínsecamente semi-supervisados) con la comparación con otros métodos enmarcados en esta taxonomía.

\clearpage
\section{Métodos implementados}

SSLTree