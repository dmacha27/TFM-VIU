\cleardoublepage

\chapter*{Resumen}
\label{resumen}
\addcontentsline{toc}{chapter}{Resumen}

En este proyecto se aborda el aprendizaje semi-supervisado, una rama del \textit{machine learning} a la que no se le da tanta importancia como al aprendizaje supervisado y no supervisado, pero que puede ofrecer una ventaja muy grande al aprovechar los datos no etiquetados. A través de una revisión bibliográfica a partir del material encontrado y también recopilado por el grupo de investigación de la Universidad de Burgos, se seleccionaron los métodos de árboles y grafos. Se ha desarrollado el método \textit{SSLTree}, basado en árboles de decisión (CART) y la teoría de \cite{levatic2017semi} y se han implementado dos algoritmos de construcción de grafos, GBILI \cite{berton2014graph} y RGCLI \cite{berton2017rgcli}, junto con el método de inferencia LGC \cite{zhou2003learning}. Los métodos han sido probados en 24 conjuntos de datos, mostrando que \textit{SSLTree} es competitivo, aunque con un rendimiento peor en \textit{ensembles}, mientras que los métodos de grafos tienen peor rendimiento de base respecto al resto de métodos comparados. Como objetivo adicional, se ha creado una aplicación web ``docente'' para la visualización de los algoritmos basados en grafos desarrollados, disponible en \url{https://dmacha.dev/gssl}.
